%************************************************
\chapter{Einleitung}\label{ch:einleitung}
%************************************************


Durch den steigenden Bedarf zur Speicherung elektrischer Energie, entsteht eine steigende Nachfrage an Neu- und Weiterentwicklungen von Batterietechnologien. Die Zink-Luft-Technik verspricht eine theoretisch dreifach größere Energiedichte als Lithium-Ion-Batterien. Die elektrochemische Reaktion von Zink mit Luftsauerstoff bietet daher ein beträchtliches Potenzial auch wiederaufladbare Speicher mit hoher Energiedichte für große Energiemengen zu bauen. Der besondere Vorteil von Zink-Luft-Batterien liegt in der hohen spezifischen Energiedichte, den niedrigen Kosten und hohen Sicherheit und bieten damit einer der vielversprechendsten Kandidaten für die Energiespeicherung. Im Rahmen eines Forschungsprojektes wird an der Fachhochschule Münster ein neuartiger Zink-Luft-Akkumulator entwickelt. Für die Entwicklung des primären Energiespeichers werden unter anderem Impedanzspektroskopien und Strömungssimulationen durchgeführt, um ein neuartiges Batterie-Management-System zu entwickeln. Für das Forschungsprojekt ist es unabdingbar, die bei der Impedanzspektroskopie anfallenden Messdaten für zukünftige Auswertungen zu speichern. Im Rahmen dieses Masterprojektes soll eine Möglichkeit geschaffen werden, um die anfallenden Messdaten aus den verschieden Versuchsreihen an den Zink-Luft-Akkumulatoren in eine Datenbank zu überführen und die gespeicherten Messdaten über ein Webinterface zu visualisieren.

\newpage


\section{Anforderungen}

Im Rahmen des Masterprojektes soll eine Möglichkeit geschaffen werden, um Messdaten aus unterschiedlichen Versuchsreihen an den oben beschriebenen Zink-Luft-Batterien über ein Webinterface zu visualisieren. Nachfolgend werden die Anforderungen im Detail beschrieben:

\begin{itemize}

\item Die Messdaten sollen in einem einheitlichen Format in einer Datenbank gespeichert werden, sodass sie für spätere Auswertungen zur Verfügung stehen. Die bislang als \code{.mat}- und \code{.csv}-Files vorliegenden Messdaten sollen automatisch in die Datenbank überführt. Zukünftig sollen Messdaten nach einem festen Messprotokoll gespeichert und kontinuierlich in die Datenbank geschrieben werden. Eine redundante Speicherung der Daten ist zu berücksichtigen.

\item Die Messdaten sollen über das Webinterface im zeitlichen Verlauf grafisch visualisiert werden. Das Interface soll eine Auswahl unterschiedlicher Messreihen sowie eine Veränderung des dargestellten zeitlichen Intervalls ermöglichen.

\item Da regelmäßig zu den Messungen aufgenommen \code{.jpg}-Bilder sollen abhängig von ihrem Zeitstempel als Thumbnail zeitlichen Verlauf der Daten mit angezeigt werden können. Klickt man auf ein Bild soll es vergrößert dargestellt werden.

\end{itemize}
